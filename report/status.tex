\chapter{Project Status}
\label{chapter:status}

This chapter presents current project progress in section \ref{section:current-progress}, challenges encountered and possible mitigation in section \ref{section:challenges}, a brief overview of next steps in section \ref{section:next-steps}, and finally the project schedule in \ref{section:schedule}.

\section{Current Progress}
\label{section:current-progress}

The project began in early September 2021. After a few weekly meetings with my supervisor and some literature review, I gained a brief understanding of the broader information security and control-flow integrity topics, and the background and methodology of this project.

By late September, the project plan was finalized and submitted, and the project website containing basic information and the project plan was set up at \url{https://wp.cs.hku.hk/fyp21015/} as required.

Between late September to early October 2021, LLVM 13.0 was successfully compiled and deployed to the system, and a short testing LLVM pass was programmed, which compiles to a shared library and is to be loaded by LLVM \texttt{opt} command to analyze the content of a single LLVM bitcode module.

In early to mid October 2021, with guidance from my supervisor, I completed the foundation of an inter-procedural analysis program, which compiles to an independent executable that is capable of loading multiple LLVM bitcode modules and analyzing them together.

Meanwhile, I have been researching on previously published papers in related areas, including control-flow integrity in general, C++ class hierarchy reconstruction and data-flow analysis.

\section{Challenges and Mitigation}
\label{section:challenges}

\subsection*{Multi-Layer Type Analysis Setup}

For results evaluation and comparison, the Multi-Layer Type Analysis approach presented in Kangjie Lu and Hong Hu's paper \cite{mlta} needs to be realized as a concrete runnable program. The authors released some partial source code at \url{https://github.com/umnsec/crix/tree/master/analyzer/src/lib} as a component in a greater project for catching bugs in operating system kernels. However, their program was written a while ago, and cannot compile with current LLVM 12.0 or LLVM 13.0.

In order to run the program and compare the results, some effort in researching and understanding LLVM API changes is needed, so the compile errors in the program can be fixed without changing the intended semantics by original authors.

\subsection*{Time and Memory Constraint in Object-Flow Analysis}

When analyzing a large code base, running time and memory consumption of the iterative search method in object-flow analysis could be a concern, especially when inter-procedural analysis is applied. A big C++ project may contain thousands of compilation units and tens or hundreds of thousands of functions and variables. It is hardly feasible to fit everything in memory during analysis. Some techniques and trade-offs may need to be made, in order to make the object-flow analysis practical in real-world use scenario.

Partitioning is a possible mitigation, that is, dividing the whole set of LLVM bitcode modules into chunks which each can be analyzed within reasonable time and memory usage. Setting a search depth and time limit threshold may also be necessary to avoid the search algorithm stalling.

\section{Next Steps}
\label{section:next-steps}

The immediate next step of the project is to fix compile errors in the Multi-Layer Type Analysis program, and extract the indirect call analysis component out for results evaluation and comparison.

Besides that, researching on C++ class hierarchy tree reconstruction from LLVM IR is also planned to be worked on next.


\section{Project Schedule}
\label{section:schedule}

Table \ref{tab:schedule} shows the tentative project schedule.

\begin{table}[h]
    \centering
    \begin{tabular}{ | m{3cm} | m{7cm}| m{1.5cm} | } 
     \hline
     Time Point & Task & Status \\
     \hline
     \hline
     September 2021 & 
     Project plan and website &
     Done
     \\[0.5cm]
     \hline
     October 2021 & 
     Research on previous work &
     In Progress
     \\[0.5cm]
     \hline
     November 2021 & 
     Preliminary system design &
     Pending
     \\[0.5cm]
     \hline
     January 2022 & 
     Submission of Intermediate Report &
     Pending
     \\[0.5cm]
     \hline
     February 2022 & 
     System implementation &
     Pending
     \\[0.5cm]
     \hline
     March 2022 & 
     Performance evaluation &
     Pending
     \\[0.5cm]
     \hline
     April 2022 & 
     Submission of Final Report &
     Pending
     \\[0.5cm]
     \hline
    \end{tabular}
    \caption{Project Schedule}
    \label{tab:schedule}
\end{table}

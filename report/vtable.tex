\chapter{Virtual Table Example}
\label{appendix:vtable}

This appendix chapter presents an example of C++ class hierarchy compiled to virtual tables and \ac{rtti} tables in LLVM IR. The example program is compiled on x86-64 LLVM 13.0.0 with compiler flags \texttt{-emit-llvm -S}.

Figure \ref{fig:vtable-src} contains a piece of C++ source code of hierarchy of 4 classes. \texttt{A} and \texttt{B} are two base classes. \texttt{C} inherits from class \texttt{B}. \texttt{D} multi-inherits from \texttt{A} and \texttt{B}.

Figure \ref{fig:vtable-llvm-ir} shows the virtual tables of class \texttt{C} and class \texttt{D}. \texttt{\_ZTV1C} is the mangled symbol name for virtual table of class \texttt{C}, and \texttt{\_ZTV1D} is the mangled symbol name for virtual table of class \texttt{D}. As introduced in Table \ref{tab:vtable-layout}, the second entry of each virtual table points to their \ac{rtti} table, \texttt{\_ZTI1C} for class \texttt{C} and \texttt{\_ZTI1D} for class \texttt{D}.

Figure \ref{fig:rtti-table-llvm-ir} shows the \ac{rtti} tables of class \texttt{C} and \texttt{D}. One can notice that the \ac{rtti} tables contains pointers to \ac{rtti} tables of their parent classes: \texttt{\_ZTI1A} and \texttt{\_ZTI1B}.

\begin{figure}[H]
    \centering
    \begin{lstlisting}[language=c++]
class A {
    public:
        virtual int f() { return 1234567; }
};

class B {
    public:
        virtual int g() { return 1234568; }
};

class C : public B {
};

class D : public A, public B {
    public:
        virtual int f() { return 1234569; }
};
    \end{lstlisting}
    \caption{C++ source code of some class hierarchy}
    \label{fig:vtable-src}
\end{figure}

\begin{figure}[H]
    \begin{lstlisting}[language=llvm]
@_ZTV1C = linkonce_odr dso_local unnamed_addr constant { [3 x i8*] } { [3 x i8*] [i8* null, i8* bitcast ({ i8*, i8*, i8* }* @_ZTI1C to i8*), i8* bitcast (i32 (%class.B*)* @_ZN1B1gEv to i8*)] }, comdat, align 8

@_ZTV1D = linkonce_odr dso_local unnamed_addr constant { [3 x i8*], [3 x i8*] } { [3 x i8*] [i8* null, i8* bitcast ({ i8*, i8*, i32, i32, i8*, i64, i8*, i64 }* @_ZTI1D to i8*), i8* bitcast (i32 (%class.D*)* @_ZN1D1fEv to i8*)], [3 x i8*] [i8* inttoptr (i64 -8 to i8*), i8* bitcast ({ i8*, i8*, i32, i32, i8*, i64, i8*, i64 }* @_ZTI1D to i8*), i8* bitcast (i32 (%class.B*)* @_ZN1B1gEv to i8*)] }, comdat, align 8
    \end{lstlisting}
    \caption{Virtual tables of class C and D in LLVM IR}
    \label{fig:vtable-llvm-ir}
\end{figure}

\begin{figure}[H]
    \begin{lstlisting}[language=llvm]
@_ZTS1C = linkonce_odr dso_local constant [3 x i8] c"1C\00", comdat, align 1
@_ZTS1D = linkonce_odr dso_local constant [3 x i8] c"1D\00", comdat, align 1

@_ZTI1C = linkonce_odr dso_local constant { i8*, i8*, i8* } { i8* bitcast (i8** getelementptr inbounds (i8*, i8** @_ZTVN10__cxxabiv120__si_class_type_infoE, i64 2) to i8*), i8* getelementptr inbounds ([3 x i8], [3 x i8]* @_ZTS1C, i32 0, i32 0), i8* bitcast ({ i8*, i8* }* @_ZTI1B to i8*) }, comdat, align 8

@_ZTI1D = linkonce_odr dso_local constant { i8*, i8*, i32, i32, i8*, i64, i8*, i64 } { i8* bitcast (i8** getelementptr inbounds (i8*, i8** @_ZTVN10__cxxabiv121__vmi_class_type_infoE, i64 2) to i8*), i8* getelementptr inbounds ([3 x i8], [3 x i8]* @_ZTS1D, i32 0, i32 0), i32 0, i32 2, i8* bitcast ({ i8*, i8* }* @_ZTI1A to i8*), i64 2, i8* bitcast ({ i8*, i8* }* @_ZTI1B to i8*), i64 2050 }, comdat, align 8
    \end{lstlisting}
    \caption{\ac{rtti} tables in LLVM IR}
    \label{fig:rtti-table-llvm-ir}
\end{figure}